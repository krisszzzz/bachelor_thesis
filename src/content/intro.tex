\section{Введение}

Графические процессоры общего назначения (GPGPU, англ. General-purpose graphics processing units) -- ускорители ...

Программные модели для гетерогенных вычислений, такие как CUDA (англ. Compute Unified Device Architecture) или OpenCL (англ. Open Computing Language), позволяют описывать поведение одного потока исполнения, который может быть мультиплицирован тысячи раз и запущен на наборе SIMD (англ. Single Instruction Multiple Data) вычислительных блоков. Такой тип архитектуры, определяемый как SIMT (англ. Single Instruction Multiple Thread), позволяет GPU ускорителю вычислять результат инструкции одновременно для нескольких потоков, объединяя их в пачки, именуемые варпы (англ. warps, wavefronts). 

Каждый такт планировщик варпов (англ. warp scheduler) выбирает, какой из набора готовых к исполнению варпов, будет исполнен следующим. Выбор оптимальной стратегии планирования, в свою очередь, является краеугольным камнем разработки высокопроизводительного GPGPU вычислителя.

Конечная цель планировщика варпов - минимизация времени исполнения кода. 