
\section{Обзор существующих решений}
TBD; \textit{Чуть более подробно}
\subsection{cuBLAS}
cuBLAS (CUDA Basic Linear Algebra Subprograms) - это высокооптимизированная библиотека линейной алгебры,
разработанная NVIDIA для своих графических процессоров. Она предоставляет реализации стандартных операций линейной алгебры такие как:
\begin{itemize}
    \item векторные операции (сложение, скалярное произведение)
    \item умножение матрицы на вектор
    \item умножение матрицы на матрицы
\end{itemize}
cuBLAS включает в себе разные подбиблиотеки, например, cuBLASLt. cuBLASLt - это облегченная версия библиотеки, представленная в
CUDA 10.1, которая предлагает более гибкий API и специализированные функции для матричных умножений.
Ключевым преимуществом cuBLAS является автоматический выбор оптимального алгоритма для конкретной операции с учетом того, на каком
устройстве будет выполняться вычисление.
Проблема библиотеки cuBLAS в том, что она является библиотекой с закрытым исходным кодом, что делает невозможным перекомпиляцию
их для запуска на функциональном и потактовом симуляторе графического процессора. Помимо этого исходный код cuBLAS представляет
большой интерес для исследования нюансов работы графического процессора и его устройства.

\subsection{cuDNN}
cuDNN (CUDA Deep Neural Network library) — это высокопроизводительная библиотека от NVIDIA, предоставляющая оптимизированные реализации примитивов
для глубокого обучения. Разработанная специально для ускорения нейронных сетей на GPU, она стала стандартом в индустрии и используется всеми
основными фреймворками машинного обучения, в том числе PyTorch.
Она предоставляет оптимизированные реализации следующих операций:
\begin{itemize}
    \item сверточные операции:
    \begin{itemize}
        \item Прямая/обратная свертка
        \item Поддержка различных форматов тензоров (NCHW, NHWC, и т.д.)
    \end{itemize}
    \item Пуллинг операции
    \item Нормализация
    \item Функции активации
\end{itemize}

Также как и cuBLASLt, cuDNN автоматически подбирает оптимальную реализацию для каждой конкретной операции.
Аналогично, cuDNN проект с закрытым исходным кодом.

