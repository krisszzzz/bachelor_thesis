\section{Введение}
Графический процессор общего назначения (далее GPGPU - general-purpose graphics processing unit) является доминирующим решением
в задачах машинного обучения. Технологии машинного обучения требуют большого количества вычислительных ресурсов и обладают высокой
степенью параллельности. Графические процессоры изначально были созданы для обработки большого количества независимых данных
(например, набора пикселей), поэтому обладают необходимой степенью параллельности и соответствуют вычислительным требованиям
методов машинного обучения.

Индустриальным стандартом GPGPU являются графические процессоры компании Nvidia \cite{nvidia_2025}. Лидирующее положение Nvidia в области
ускорителей для искусственного интеллекта является следствием не только высокой производительности аппаратных решение, но также
развитой системы программного обеспечения. Среди программного стека Nvidia наиболее важными частями являются:
\begin{itemize}
    \item Программно-аппаратная модель CUDA \cite{cuda_guide_2025}
    \item Специализированные оптимизированные библиотеки, такие как:
    \begin{itemize}
        \item cuBLAS \cite{cublas_2025} — для операций линейной алгебры
        \item cuDNN \cite{cudnn_2025} — для операций свертки, пакетной нормализации и т.д.
    \end{itemize}
    \item интеграция с популярными фреймворками машинного обучения, такими как \\
          PyTorch \cite{pytorch_2025} и TensorFlow \cite{tensorflow_2025}
\end{itemize}
Такая сильно развитая система программного обеспечения позволяет разработчикам использовать эффективные инструменты для реализации задач машинного
обучения на GPGPU.

При разработке конкурентноспособных решений в области GPGPU необходимо уделить особое внимание разработке эффективных
программных инструментов и библиотек, а также обеспечить совместимость с существующими фреймворками машинного обучения.
Помимо достижения высокой производительности, это обеспечит плавный переход разработчиков на новую платформу.

Настоящая дипломная работа посвящена разработке высокопроизводительных библиотек машинного обучения, предназначенных для работы на GPGPU, поддерживающую
программно-аппаратную модель CUDA. В качестве исследовательской базы используется функциональный и потактовый симулятор графического процессора,
поддерживающий модель CUDA, на основе симулятора компьютерных систем gem5 \cite{gem5_2025}. Основное внимание уделено разработке специализированных библиотек:
линейной алгебры cuBLAS и нейронных сетей cuDNN, а также адаптации этих инструментов для совместимости c фреймворком машинного обучения PyTorch.
Симулятор вместе с программным обеспечением, включающая в себе драйвер симулируемого графического процессора и библиотеку среды выполнения CUDA позволяет
запускать CUDA-программы. Ввиду значительных ограничений производительности программной симуляции графического процессора, исследование
производительности было проведено на реальных графических процессорах Nvidia: Tesla V100 \cite{tesla_v100_2025} и Jetson Xavier \cite{jetson_xavier_2025}.

Специализированные библиотеки, такие как cuBLAS и cuDNN имеют решающее значение при разработке программного обеспечения, необходимого для успешной
работы задач машинного обучения на GPGPU, поэтому исследование этих инструментов стало основным. В ходе исследования были выбраны наиболее критически
важные интерфейсы библиотек cuBLAS и cuDNN:
\begin{itemize}
    \item \texttt{cublasLtMatmul} — общее матричное умножение (GEMM — general matrix multiplic\-ation)
    \item \texttt{cudnnConvolutionForward} — прямая свертка тензора (обычно 4-мерного в формате NCHW или NHWC)
    \item \texttt{cudnnBatchNormalizationForwardInference} — пакетная нормализация тензора
\end{itemize}
Ввиду того, что библиотеки cuBLAS и cuDNN являются проприетарными решениями, необходима собственная реализация этих интерфейсов.
Данные интерфейсы были разработаны, и было проведено исследование их производительности на графических процессорах Nvidia. Реализации данных интерфейсов
активно задействуют библиотеку CUTLASS \cite{cutlass_2025}, представляющую собой высокопроизводительную коллекцию шаблонов и компонентов для линейной алгебры
на CUDA и являющаяся проектом с открытым исходным кодом. Основными преимущества CUTLASS являются:
\begin{itemize}
    \item Возможность переиспользования частей высокопроизводительных алгоритмов
    \item Настройка параметров алгоритмов
    \item Поддержка различных типов данных
\end{itemize}

Для верификации корректности работы реализованных интерфейсов была проведена серия экспериментов по запуску реальных нейронных сетей с использованием фреймворка PyTorch.
Тестирование включало:
\begin{itemize}
    \item Сравнение результатов вычислений с эталонными реализациями
    \item Замеры производительности на различных типах нейронных сетей
    \item Анализ масштабируемости при работе с тензорами разных размеров и форматов
\end{itemize}

Были получены результаты, демонстрирующие работоспособность предложенных реализаций и их соответствие требованиям высокой производительности.